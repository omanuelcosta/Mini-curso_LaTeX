\documentclass[12pt]{article}
\usepackage{graphicx}
\usepackage{hyperref}
\usepackage[Brazil]{babel}

\begin{document}

Esta é a Figura \ref{meme-1}.

\begin{figure}[!ht]
	\centering
	\includegraphics[width=0.5\textwidth]{meme_superior.png}
	\caption{A \LaTeX\ figure.}\label{meme-1}
\end{figure}

Lorem ipsum dolor sit amet, consectetur adipiscing elit. Morbi et lectus ullamcorper, eleifend lorem eu, euismod dolor. Donec auctor tortor a mi commodo, sed gravida libero accumsan. Nunc ante arcu, imperdiet feugiat nisl ac, gravida ornare tellus. Vestibulum ante ipsum primis in faucibus orci luctus et ultrices posuere cubilia curae; Sed sagittis pretium hendrerit. Donec euismod eros et augue porttitor rhoncus. Vestibulum dignissim diam eget felis hendrerit sodales. Nam viverra interdum eleifend. Donec nec cursus neque. Mauris dictum eleifend imperdiet. In viverra blandit lectus, sit amet congue libero euismod volutpat. Nulla a magna pharetra est pretium tempor a imperdiet nisl. Aqui temos as figuras \ref{meme}, \ref{meme1} e \ref{meme2}.

\begin{figure}[!ht]
	\centering
	\includegraphics[width=0.5\textwidth]{meme_superior.png}
	\caption{A \LaTeX\ figure.}\label{meme}
\end{figure}

\begin{figure}[!ht]
	\centering
	\includegraphics[width=0.5\textwidth]{meme_superior.png}
	\caption{A \LaTeX\ figure.}\label{meme1}
\end{figure}

Esta é a Tabela \ref{tabelinha}.

\begin{table}
	\centering
	\begin{tabular}[c]{l c c r}
		\hline
		célula 1 & cel 2 & cel 3 & cel 4 \\
		célula 1 & cel 2 & cel 3 & cel 4 \\
		célula 1 & cel 2 & cel 3 & cel 4 \\
		\hline
	\end{tabular}
	\caption{Tabelinha}\label{tabelinha}
\end{table}

\begin{figure}[!ht]
	\centering
	\includegraphics[width=0.5\textwidth]{meme_superior.png}
	\caption{A \LaTeX\ figure.}\label{meme2}
\end{figure}

\end{document}
