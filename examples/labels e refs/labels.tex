\documentclass[12pt]{article}
\usepackage{graphicx}

\begin{document}

\begin{figure}
	\centering
	\includegraphics[width=0.6\textwidth]{meme_superior.png}
	\caption{Uma figura \LaTeX}
	\label{fig:meme}
\end{figure}

\begin{equation}
	E = m c ^2
\end{equation}

\begin{equation}
	E^2 = (m c^2)^2 + (p c)^2
	\label{eqn:einsteincompleta}
\end{equation}
\begin{table}
	\centering
	\begin{tabular}{l c c r}
		\hline
		amostra & $\delta_{ij}$ & $\alpha_{ij}$
			& tempo (s) \\
		\hline
		1 & 23.1 & 32.4 & 16 \\
		2 & 42.1 & 65.4 & 17 \\
		3 & 55.1 & 82.4 & 23 \\
		\hline
	\end{tabular}
	\caption{Uma tabela \LaTeX}
	\label{tab:alpha_delta}
\end{table}

A Figura \ref{fig:meme} está na página \pageref{fig:meme}. Já a Equação \ref{eqn:einsteincompleta} foi usada para produzir a Tabela \ref{tab:alpha_delta} que está na página \pageref{tab:alpha_delta}.

\end{document}